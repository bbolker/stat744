\documentclass{article}

\addtolength{\hoffset}{-2cm}
\addtolength{\textwidth}{3.5cm}
\addtolength{\voffset}{-2cm}
\addtolength{\textheight}{3cm}

\usepackage{amsmath}		 	
\usepackage{amsfonts}	 	
\usepackage{graphicx}			
\usepackage{fancyhdr}	 		
\usepackage{hyperref}
\pagestyle{fancy}
\chead{Extended Kalman Filter - Lab}
\rhead{}

\begin{document}
\section{EKF - Tracking}
A logistic growth model with rate $r$ and carrying capacity $k$ can be written as
$$\frac{dp}{dt}=rp\left( 1-\frac{p}{k}\right)$$
with initial guess $p_0$, the logistic growth model can be analytically solved as.
$$p=\frac{kp_0exp(rt)}{k+p_0(exp(rt)-1)}$$
A population data is modeled based on the analytical solution of logistic growth model with additive random process error
$$p_{t}=\frac{kp_{t-1}exp(r \Delta t)}{k+p_{t-1}(exp(r \Delta t)-1)} + v$$
An extended Kalman Filter is used to track the population given that the variance of process and observation error is known. state space is assumed as $x=[r \text{  }  p]^T$ and observation model is given as
$$z=[0\text{  }  1][r\text{  }  p]^T+w$$
In the given code the sample data is  synthetically generated and tracking algorithm is implemented.

\subsection{Exercise 1}
In the sample code only population in the state space is assumed to have additive process error. Modify the code such that rate of population growth also has additive process error. 
\subsection{Exercise 2}
Implement Kalman filter tracking algorithm for the same logistic population growth by using Euler's explicit time stepping scheme.
$$p_t=p_{t-1}+\Delta t \left( rp_{t-1}\left( 1-\frac{p_{t-1}}{k}\right)+ v\right )$$

\section{EKF - Parameter Estimation}
A sample code for estimating parameter a in the equation $$y=a^2x^2+x+1$$ is given in the sample code.  The data is synthetically generated to observe the error of estimated parameter form the true value of the parameter.
\subsection{Excercise 3}
Estimate the initial velocity ($v_0$) of the projectile for the given data in projectile.txt using initial angle as $\theta = pi/4$. Projectile motion equation is 
$$x=v_0t\cos(\theta)$$
$$y=v_0t\sin(\theta)-\frac{gt^2}{2}$$
\end{document}